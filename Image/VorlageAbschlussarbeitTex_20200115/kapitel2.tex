\chapter{Beispiele}
\label{beispiele}

Das hier ist das Kapitel Beispiele ...

\section[$e^{x}$ - Beispiel]{\boldmath{$e^{x}$} - Beispiel}
\label{e_hoch_x_beispiel}

...mit einem Abschnitt in dem eine Formel geschrieben wurde ...

\subsection{Jetzt aber wirklich}
\label{jetzt_aber_wirklich}

... mit noch einem folgenden Unterabschnitt. "`Tiefer"' sollte man jedoch in den Kapiteln nicht gehen.\newline

Hier folgen nun (wirklich) die Beispiele mit den wichtigsten Elementen.

Hier kommt ein Bild im pdf-Format:
\begin{figure}[ht]
 \begin{center}
  \includegraphics[width=0.8\textwidth]{bilder/TU_BS_IMAB.pdf}
  \caption{Logo des IMAB TU Braunschweig}
  \label{fig:logo_IMAB}
 \end{center}
\end{figure}

Und hier ein Beispiel für eine Tabelle.
\begin{table}[ht]
 \begin{center}
 \caption{Verwendete Matrizen}
 \label{tab:matrizen}
  \begin{tabular}{|l|c|c|}
   \hline
   Matrix & Dimension & Symbol \\
   \hline
   Systemmatrix & $n \times n$ & ${\textbf A}$  \\
   \hline
   Ausgangsmatrix & $m \times n$ & ${\textbf C}$  \\
   \hline
  \end{tabular}
 \end{center}
\end{table}

Und wieder ein Bild:
\begin{figure}[ht]
	\centering
	 \includegraphics[width=0.4\textwidth]{bilder/motor.jpg}
	\caption{Abbildung eines Elektromotors}
	\label{fig:motor}
\end{figure}

Jetzt kommt mal Mathematik\footnote{Man beachte: Es gibt viele Möglichkeiten Formeln darzustellen. Für nähere Informationen siehe bitte die amsmath-Dokumentation.} an die Reihe:
\begin{equation}
\begin{split}
		 K(s) &= V_R \cdot \frac{1}{T_I \cdot s}\cdot \left( T_I \cdot s + 1 \right) \\
					  &= V_R \cdot \left( 1 + \frac{1}{T_I \cdot s} \right) = \underbrace{V_R}_{P-Anteil} + \underbrace{\frac{V_R}{T_I \cdot s}}_{I-Anteil} \\
						&\text{mit $V_R$ : Reglerverstärkung} 
\end{split}
\end{equation}

Hier nehmen wir Bezug auf Abbildung \ref{fig:motor} auf Seite \pageref{fig:motor}: Dort sieht man einen Elektromotor.\newline
Nachdem dieser Zeilenumbruch durchgeführt wurde, kommen jetzt noch andere Sachen an die Reihe\footnote{So z.~B. die Verwendung von Fußnoten.}. Daneben aber auch die Anwendung von \textit{kursiver} und \textbf{fettgedruckter} Schrift. Auch das paraphrasierte Zitieren ist nicht zu verachten\cite{PROFMUSTER11}.

Weil das hier alles so komplex ist, kommt jetzt eine Aufzählung:
\begin{enumerate}
	\item Das passiert als erstes.
	\item Und das hier als zweites.
	\item ... und als drittes das hier.
\end{enumerate}
Oder wie wäre es mit einer Auflistung?
\begin{itemize}
	\item Asynchronmaschine
	\item Synchronmaschine
	\item Gleichstrommaschine
\end{itemize}
Bei machen Arbeiten macht es evtl. auch Sinn, Quelltext in den Fließtext einzufügen, um Sachverhalte zu verdeutlichen. 
\begin{verbatim}
	function my_function(input:string; var output: string):Integer;
\end{verbatim}