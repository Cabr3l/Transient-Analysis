\addchap{Abstract}
\label{abtract} 

In this thesis, a bidirectional inverter from a previous master thesis [5] is investigated. Errors as well as potentials for improvement will be documented. On the basis of these investigations, a further development of the inverter will be carried out, which can be used both as a primary and secondary converter in a resonant converter. It should be able to transmit an active power of up to 20 kW. For this purpose, DC link voltages between 300 V and 800 V and currents up to 67 A (RMS) are to be endured by the inverter without any problems.

In the basic chapter, the functionality of a resonant converter is explained. Different configurations of the resonant circuit are presented and explained. Among others the SS- and the LLC-resonant circuit. The structure and the operation mode of a full bridge are also described. The focus is put here on SiC as semiconducting material for electrical switches. A comparison between SiC-MOSFET and Si-MOSFET is also subject of the basic chapter.

The designed circuits for the further development of the inverter (power electronics, measurement circuits) are then presented and explained. Subsequently, the constructed inverter is put into operation. The results of the commissioning are documented. Suggestions for improvement are also presented at the end.

\vspace{1cm}

\textbf{Keywords:}
\begin{itemize}
 \item Bidirectional inverter for inductiv power transfer
 \item Master thesis at IMAB
 \item Resonant-converter und Resonant-coil
\end{itemize}




